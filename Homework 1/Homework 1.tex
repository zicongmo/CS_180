\documentclass[12 pt, oneside]{article}
\usepackage{geometry}
\usepackage{amsmath}
\usepackage{listings}
\geometry{letterpaper, margin = 1 in}

\lstset{frame=tb, 
	language=Python,
	aboveskip=3mm,
	belowskip=3mm,
	showstringspaces=false,
	columns=flexible,
	breaklines=true,
	basicstyle={\small\ttfamily},
	numbers=none,
	breakatwhitespace=true,
	tabsize=4
}

\title{Homework 1}
\author{Zicong Mo (UID: 804654167)}
\date{January 19th, 2017}
\begin{document}
\maketitle
\begin{enumerate}
\setcounter{enumi}{1}
\item 
Since the computer can compute $10^{10}$ operations per second and there are 3600 seconds in one hour, the computer can compute a total of $10^{10} \times 3600 = 3.6 \times 10^{13}$ operations. To get the largest input size $n$ for which we would get the result within an hour, we solve the inequality $f(n) \le 3.6 \times 10^{13}$, and take the largest integer $n$ for which the inequality is true.
\begin{enumerate}
\item $n^2 \le 3.6 \times 10^{13} \Rightarrow n \le 6 \times 10^6$. The largest input size is $n = 6 \times 10^6$.
\item $n^3 \le 3.6 \times 10^{13} \Rightarrow n \le 33019.27$. The largest input size is $n = 33019$.
\item $100 n^2 \le 3.6 \times 10^{13} \Rightarrow n \le 6 \times 10^5$. The largest input size is $n = 6 \times 10^5$.
\item $n \log_2 n \le 3.6 \times 10^{13} \Rightarrow n \le 9.06 \times 10^{11}$. The largest input size is $n = 9.06 \times 10^{11}$.
\item $2^n \le 3.6 \times 10^{13} \Rightarrow n \le 45.03$. The largest input size is $n = 45$.
\item $2^{2^n} \le 3.6 \times 10^{13} \Rightarrow n \le 5.49$. The largest input size is $n = 5$. 
\end{enumerate}
\clearpage
\item
\begin{align*}
f_2(n) &= \sqrt{2n} \\
f_3(n) &= n + 10 \\
f_6(n) &= n^2 \log n \\
f_1(n) &= n^{2.5} \\
f_4(n) &= 10^n \\
f_5(n) &= 100^n
\end{align*}
\clearpage
\setcounter{enumi}{6}
\item 
A song with $n$ total words and $c$ words per line has a total of $n/c$ lines. Suppose we have just finished iteration $k$ of the song. The next line is the new addition to the song, before repeating iteration $k$ again.  Therefore the number of lines at the end of iteration $k$ is the $k$th triangular number. The total number of iterations can then be solved with the equation:
$$\frac{k(k+1)}{2} = \frac{n}{c} $$
$$k = \frac{1}{2}(\sqrt{1 + \frac{8n}{c}} -1) $$
An example of the algorithm in Python:
\begin{lstlisting}
# Assume the words are in a list of length n
num_iterations = 0.5 * (sqrt(1 + 8*n/c) - 1)
for i in range(num_iterations):
	line_number = i * (i+1) /2
	for j in range(c):
		print(words[line_number*c+j], end=" ")
	print("\n")
\end{lstlisting}
For each of the $k$ iterations, we print out $c$ words. Therefore the complexity of this algorithm is $O(kc) = O(c\sqrt{\frac{n}{c}})$.
\clearpage
\item
\begin{enumerate}
\item
Let $x$ be the smallest value such that $x(x+1) \ge 2n$. \\
Start by dropping a jar at rung $x$. \\
If it breaks, then we can do a linear search with the other jar from rung 1 to rung $x-1$. We can find exactly which of these is the highest safe rung in at most $1 + x - 1 = x$ drops. \\
Otherwise, drop a jar at rung $x + (x-1)$. If this jar breaks, then we can do a linear search from rung $x+1$ to rung $2x -2$. The highest safe rung in this range can be found in at most $2 + (2x-2) - (x+1) + 1 = x$ drops. \\
If the jar didn't break, continue on at rung $x + (x-1) + (x-2)$, and so on. \\
\par We show that the number of steps needed to find the highest safe rung is at most $x$. To do this, we prove that every rung is reached in at most $x$ drops. Note that the first $x$ rungs are covered by at most $1 + (x-1) = x$ drops, the next $x-1$ rungs are covered by at most $2 + (2x-2) - (x+1) + 1 = x$ drops, the next $x-2$ rungs are covered by at most $3 + (3x-4) - 2x + 1 = x$ drops, etc.\\
Because we define $x$ as the smallest number such that $x + (x-1) + ... + 1 \ge n$, the entire ladder can be covered by at most $x$ drops. \\
\par Since the drop-time of this algorithm is at most $x$, and the value of $x$ depends on $\sqrt{n}$, our algorithm runs in $O(\sqrt{n})$, which is better than linear time.
\item 
We apply a similar strategy in the general case. \\
Let $x$ be the smallest integer greater than $n^{1 - \frac{1}{k}}$. Let $f_k(n)$ be the minimum number of drops needed to locate the highest safe rung using this strategy in a ladder with $n$ rungs and $k$ jars. We drop the first jar at integer multiples of $x$, until we reach the top rung or the jar breaks at some lower rung. \\
Because $x \ge n^{1 - \frac{1}{k}}$, note that 
$$ nx \ge n \cdot n^{1 - \frac{1}{k}}$$
$$ \frac{n}{n^{1 - \frac{1}{k}}} \ge \frac{n}{x} $$
Because we drop the first jar at integer multiples of $x$, the number of times we can drop the first jar is bounded by $n/x$: 
$$d_1 \le n/x \le n/n^{1 - \frac{1}{k}} = n^{\frac{1}{k}}$$
Once the first jar breaks, we have $k-1$ jars to search through $x$ floors, which can be done in $f_{k-1}(x)$ drops. This recursion can be written as:
\begin{equation*}
\begin{split}
f_k(n) & = d_1+ f_{k-1}(x) \\
& \le n^{\frac{1}{k}} + f_{k-1}(x) \\
	& \le  n^{\frac{1}{k}} + f_{k-1}(n^{1 - \frac{1}{k}})
\end{split}
\end{equation*}
We guess that the runtime of our algorithm is bounded by some polynomial function whose degree decreases as $k \rightarrow \infty $, as this type of function gives us the desired limiting behavior of $f$. Because of the prevalence of $n^{1/k}$ in our recurrence, we guess that, for some value $C_k$,
$$ f_k(n) \le C_k n^{1/k}. $$
Assume that $f_{k-1}(n) \le C_{k-1}n^{1/(k-1)} \Rightarrow f_{k-1}(n^{1 - \frac{1}{k}}) \le C_{k-1}(n^{1 - \frac{1}{k}})^{1/(k-1)}$. Then, by induction,
\begin{equation*}
\begin{split}
f_k(n) & \le n^{\frac{1}{k}} + f_{k-1}(n^{1 - \frac{1}{k}})\\
& \le n^{\frac{1}{k}} + C_{k-1}(n^{1 - \frac{1}{k}})^{1/(k-1)}\\
& \le n^{\frac{1}{k}} + C_{k-1}(n^{\frac{k-1}{k}})^{1/(k-1)}\\
& \le n^{\frac{1}{k}} + C_{k-1}n^{\frac{1}{k}}\\
& \le n^{\frac{1}{k}}(1 + C_{k-1})
\end{split}
\end{equation*}
Since we want $ f_k(n) \le C_k n^{1/k} $, if we can solve $1 + C_{k-1} = C_k$, our inductive step is complete. We see from inspection that $C_k = k$ has this property.
\begin{equation*}
\begin{split}
f_k(n) & \le n^{\frac{1}{k}}(1 + C_{k-1})\\
 & \le n^{\frac{1}{k}}(1 + k-1)\\
& \le n^{\frac{1}{k}}(k)\\
& \le C_kn^{\frac{1}{k}}
\end{split}
\end{equation*}
To finish the proof, we show that this inequality holds for the base case $k = 2$. We showed in part (a) that this strategy produces a runtime of $O(n^{1/2})$, so the runtime must also be bounded by $2n^{1/2}$. Therefore we have shown that the maximum number of drops needed by this strategy to find the critical rung is 
$$ f_k(n) \le kn^{1/k} $$
Note that 
$$ \lim_{n \rightarrow \infty} \frac{f_{k+1}(n)}{f_k(n)} = \lim_{n \rightarrow \infty} \frac{k+1}{k} \frac{n^{1/(k+1)}}{n^{1/k}} = \lim_{n \rightarrow \infty} \frac{k+1}{k} n^{-\frac{1}{k(k+1)}} = 0,$$
so each function $f_{k+1}$ grows slower than the previous function $f_k$.
\end{enumerate}
\end{enumerate}
\end{document}