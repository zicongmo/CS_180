\documentclass[oneside, 12pt]{article}
\usepackage{geometry}
\usepackage{amsmath}
\usepackage{amssymb}
\usepackage{enumitem}
\usepackage{graphicx}
\geometry{margin=1in}

\begin{document}
\begin{enumerate}
\setcounter{enumi}{10}
\item	
Suppose we are given a value of $b > 1$. Since they claim that the Forward-Edge-Only Algorithm is guaranteed to find a flow of at least $1/b$ regardless of the graph, it suffices to construct a graph for which the algorithm fails to find a flow of this kind.

Let $G = (V,E)$ be a graph with $4b + 2$ vertices, with source $s$ and sink $t$. There are $2b$ vertices $i_1, i_2, ..., i_{2b}$, each of which are connected to the source $s$. Similarly, there are $2b$ vertices $j_1, j_2,...,j_{2b}$, each of which are connected to the sink $t$. Furthermore, $i_1$ is connected to $j_1$, $i_2$ is connected to $j_2$, and so on. In addition, we add an edge from $j_1$ to $i_2$, $j_2$ to $i_3$, and so on. Let the capacity of each edge be $c$. 

Because they clalim that their algorithm works regardless of how it chooses its forward-edge paths, we can choose a forward-edge path for this graph. Note that the path $s \rightarrow i_1 \rightarrow j_1 \rightarrow i_2 \rightarrow ... \rightarrow j_{2b} \rightarrow t$ is a path that uses only the forward-edges, and in fact passes through each of the vertices. However, because this algorithm deletes the backwards edges, there is no longer a path from $s$ to $t$, since all the edges between $i$ and $j$ are fully filled. Therefore, the algorithm terminates, with a maximum flow of $c$. 

But there exists a maximum flow of $2bc$, since instead of following the edge $(j_1, i_2)$, we could have simply taken the edge $(j_1, t)$ and made it to the sink. Since there are $2b$ vertices, the maximum flow is $2bc$. Therefore the algorithm finds a flow of value $c/2bc = 1/2b < 1/b$ times the maximum, so their claim is false.
\clearpage
\setcounter{enumi}{13}
\item
\begin{enumerate}
\item
In order to use the max-flow algorithm, we need a source node $s'$ and a sink node $t'$. We create a new node $s'$, and connect $s'$ to each of the vertices $x \in X$. Similarly, we add a new node $t'$ to the graph, and connect each of the vertices $s \in S$ to $t'$. Note that the problem is asking to find edge-disjoint paths from $s'$ to $t'$. By Theorem 7.44, we can use Ford-Fulkerson to find a maximum set of edge-disjoint $s'-t'$ paths $P$ in $O(nm^2)$ time. Since we want an escape route from every $x \i X$ to some safe node, we can simply evaluate the cardinality of the set of edge-disjoint paths returned by Ford-Fulkerson. If $|P| < |X|$, then there is a vertex in $X$ for which there is no escape route. 

Finding $P$ requires $O(nm^2)$ time, and evaluating the cardinality of $P$ takes linear time. Therefore, the algorithm runs in polynomial time.
\item
Note that node-disjoint paths are also edge-disjoint, as two paths cannot share an edge $(u, v)$ if they are not allowed to share $u$ or $v$. Therefore, it suffices to find $|X|$ node-disjoint paths from $s'$ to $t'$. For each node $v$ in $G$, we split $v$ into two nodes $v_{in}$ and $v_{out}$, and add an edge of capacity 1 from $v_{in}$ to $v_{out}$. We replace every edge $(u, v)$ to be an edge from $(u_{out}, v_{in})$ with capacity 1 instead. We run the max-flow algorithm from $s'_{out}$ to $t'_{in}$. 

Note that the division of each node from $v$ to $v_{in}, v_{out}$ causes the node-disjoint condition. To see this, consider two flows that pass through the same node in $G$ before the node division occurs. Each of these flows has capacity 1. But after we divide the node, the edge from $v_{in}$ to $v_{out}$ only has capacity 1, meaning that both of these flows cannot pass through this same edge. At most one flow can go through each $(v_{in}, v_{out})$ pair. Therefore at the conclusion of the algorithm, the value of the maximum flow is the number of node-disjoint paths from $s$ to $t$. If the maximum flow is fewer than the cardinality of $X$, then there is a vertex in $X$ for which there is no escape route.

Dividing each of the nodes takes $O(n)$ time, and running Ford-Fulkerson on the modified graph takes $O(nm^2)$. Therefore, the algorithm runs in polynomial time.

As an example of a graph with a set of edge-disjoint escape routes but not node-disjoint escape routes, consider the following graph, with $V= \{1, 2, 3, 4, 5\}$ and $E = \{(1, 3), (2, 3), (3, 4), (3, 5) \}$. Nodes $1$ and 2 are the populated nodes, and nodes $4$ and 5 are the safe nodes. Note that there is a edge-disjoint path from 1 to 4, and 1 to 5. However, as both must pass through node 3, these paths are not node-disjoint.
\end{enumerate}
\clearpage
\setcounter{enumi}{16}
\item
Because the maximum flow in the network before any edges are severed is $k$, we know that the min-cut has capacity $k$. Therefore, there exists some partition of vertices into $A, B$ such that the sum of the flow leaving $A$ minus the sum of the flow entering $A$ is $k$. Consider each of the $k$ edges that the attacker has destroyed. Since destroying these edges causes a minimum cut to appear (under the assumption that destroying $k$ edges is the minimum number needed to separate $s$ from $t$), our partition $A, B$ is exactly the partition caused by a min-cut.

Because the max-flow is $k$, we know that there are at least $k$ edges leaving $s$ with flow 1. Therefore, using Ford-Fulkerson, we can find a set of $k$ edge-disjoint paths from $s$ to $t$ (assuming that the edges haven't been severed yet). For any given path $P = \{v_1, ..., v_i\}$, we can binary search with ping in order to find the edge that causes a failure. Because each path can only go through each node once, we know that $i \le n$. Therefore, it takes $O(\log n)$ pings to binary search for a edge for any given path. Since we have $k$ paths, we require $O(k\log n)$ pings in order to find all of the breaks.
\clearpage
\setcounter{enumi}{28}
\item
Construct a directed graph $G = (V, E)$, with each of the $n-1$ software applications $(2, ...,n)$ representing a node. We let node 1 be the source node, and let an arbitrary node $t$ be the sink node. We split each application into two separate nodes $i$ and $i'$, with $i$ representing moving the application to a new system, and $i'$ representing not moving the application to a new system. We connect node 1 to all $i$, and connect node $i'$ with $t$. Additionally, we connect $j'$ to $i$ for every $ i \ne j$. 

For each edge $(1, i)$, we let the capacity of this edge be $b_i - x_{1i}$, since we already know that application $1$ will be staying on the initial server. For each edge $(i, t)$, we let the capacity of this edge be $\infty$, since we don't want the sink to become the bottleneck. For each edge $(i, j'), i\ne j$, we let the capacity of this edge be $b_i - x_{ij}$, representing the expense of taking $i$ but not $j$. Finally, we run the Ford-Fulkerson on this graph to get the maximum flow. We follow the flow in the diagram, and if the flow passes through node $i$, we say that we port application $i$ over. Otherwise, if it passes through node $i'$, we say that we don't port application $i$ over.

Constructing the nodes in the graph takes $O(n)$ time, since each application must be split into two nodes. Constructing the edges in the graph takes $O(n^2)$ time, since there is an edge for each pair of applications $(i, j')$. Finally, running Ford-Fulkerson and tracing through the flow at the end takes $O(nm^2)$ time. Therefore, this algorithm runs in $O(nm^2)$. 
\end{enumerate}
\end{document}